\usepackage{xstring}
\usepackage{pgffor}

\ExplSyntaxOn
\NewDocumentCommand{\extractlast}{O{.}mo}
 {
  \seq_set_split:Nnn \l_stroobants_string_seq { #1 } { #2 }
  \IfNoValueTF { #3 }
   {
    \seq_item:Nn \l_stroobants_string_seq { -1 }
   }
   {
    \tl_set:Nx #3 { \seq_item:Nn \l_stroobants_string_seq { -1 } }
   }
 }
\seq_new:N \l_stroobants_string_seq
\ExplSyntaxOff

\newbool{jdocRefTypewriter}
\newbool{jdocRefShowPackage}
\newbool{jdocRefShowField}
\newbool{jdocRefShowMethod}
\newbool{jdocRefShowParameters}
\pgfkeys{
	/jdocRef/.is family,
	/jdocRef/.cd,
	full text/.estore in = \jdocRefFullText,
	parameters text/.estore in = \jdocRefParametersText,
	base url/.estore in = \jdocRefBaseUrl,
	parameters style/.estore in = \jdocRefParametersStyle,
	typewriter/.is if = jdocRefTypewriter,
	show package/.is if = jdocRefShowPackage,
	show field/.is if = jdocRefShowField,
	show method/.is if = jdocRefShowMethod,
	show parameters/.is if = jdocRefShowParameters,
	default/.style = {
		full text =,
		parameters text=,
		typewriter = true,
		show package = false,
		show field = true,
		show method = true,
		show parameters = false,
	},
}

%Thx https://tex.stackexchange.com/a/34318.
%Approach: define the following jdocRef macros. AtSign (starts with at); Module (before unique / or empty); Method (after unique # if has ( or empty); Field (after unique # if no ( or empty if no #); FQName (between / and # or after / or before # or everything); FQNameSlashes (FQName with slashes instead of dots); Class (between last dot and #).
%The main argument specifies everything needed to build the URL and the default appearance (@ or no @…), except possibly for default global variables such as base url. The default appearance can be changed with keys.
%Notation: I considered using double colon to separate class from method (https://stackoverflow.com/a/59418704), and writing parameters as, for example, Random::doubles(origin: double, bound: double), following UML notation. But to be coherent, must then use the double colon to separate package elements and class from inner classes, for example, java::util::Random and HTML::Attribute::HEIGHT, which is not readable. Seems more reasonable to follow closely the Javadoc notation (after all, it’s Java classes that we refer to, not language-neutral classes). The only addition being when I need the parameter names, which I write as Random#doubles(double origin, double bound).
\newcommand{\jdocRef}[2][]{%
	\pgfkeys{/jdocRef/.cd, default, #1}%
	\edef\jdocRefFull{#2}%
	\IfBeginWith{#2}{@}{%
		\edef\jdocRefAtSign{@}%
		\StrBehind{#2}{@}[\jdocRefWithoutAt]%
	}{%
		\edef\jdocRefAtSign{}%
		\edef\jdocRefWithoutAt{#2}%
	}%
	\IfSubStr{#2}{/}{%
	}{%
		\typeout{Missing slash!}
	}%
	\StrBefore{\jdocRefWithoutAt}{/}[\jdocRefBeforeFirstSlash]%
	\StrBehind{\jdocRefWithoutAt}{/}[\jdocRefAfterFirstSlash]%
	\IfSubStr{\jdocRefAfterFirstSlash}{/}{%
		\edef\jdocRefModule{\jdocRefBeforeFirstSlash}%
		\StrBefore{\jdocRefAfterFirstSlash}{/}[\jdocRefPackage]%
		\StrBehind{\jdocRefAfterFirstSlash}{/}[\jdocRefClassAndRest]%
	}{%
		\edef\jdocRefModule{}%
		\edef\jdocRefPackage{\jdocRefBeforeFirstSlash}%
		\edef\jdocRefClassAndRest{\jdocRefAfterFirstSlash}%
	}%
	\StrSubstitute{\jdocRefPackage}{.}{/}[\jdocRefPackageSlashes]%
	\IfSubStr{#2}{\#}{%
		\StrBehind{#2}{\#}[\jdocRefAfterSharp]%
		\IfSubStr{\jdocRefAfterSharp}{(}{%
			\StrBehind{#2}{\#}[\jdocRefMethodWithParameters]%
		}{%
			\edef\jdocRefMethodWithParameters{}%
		}%
	}{%
		\edef\jdocRefMethodWithParameters{}%
	}%
	\StrSubstitute{\jdocRefMethodWithParameters}{ }{}[\jdocRefMethodWithParametersWithoutSpaces]%
	\StrBefore{\jdocRefMethodWithParameters}{(}[\jdocRefMethodWithoutParameters]%
	\IfSubStr{#2}{\#}{%
		\StrBehind{#2}{\#}[\jdocRefAfterSharp]%
		\IfSubStr{\jdocRefAfterSharp}{(}{%
			\edef\jdocRefField{}%
		}{%
			\StrBehind{#2}{\#}[\jdocRefField]%
		}%
	}{%
		\edef\jdocRefField{}%
	}%
	\IfSubStr{\jdocRefClassAndRest}{\#}{%
		\StrBefore{\jdocRefClassAndRest}{\#}[\jdocRefClass]%
	}{%
		\edef\jdocRefClass{\jdocRefClassAndRest}%
	}%
	%
	%\jdocRefPrintAll%
	%
	\edef\jdocRefTarget{\jdocRefBaseUrl}%
	\IfEq{\jdocRefModule}{}{%
	}{%
		\appto\jdocRefTarget{\jdocRefModule}%
		\appto\jdocRefTarget{/}%
	}%
	\appto\jdocRefTarget{\jdocRefPackageSlashes}%
	\appto\jdocRefTarget{/}%
	\appto\jdocRefTarget{\jdocRefClass}%
	\appto\jdocRefTarget{.html}%
	\IfEq{\jdocRefField}{}{%
	}{%
		\appto\jdocRefTarget{\#\jdocRefField}%
	}%
	\IfEq{\jdocRefMethodWithParametersWithoutSpaces}{}{%
	}{%
		\IfEq{\jdocRefParametersStyle}{hyphen}{%
			\StrSubstitute{\jdocRefMethodWithParametersWithoutSpaces}{(}{-}[\jdocRefMethodWithParametersWithoutSpacesWithHyphens]%
			\StrSubstitute{\jdocRefMethodWithParametersWithoutSpacesWithHyphens}{,}{-}[\jdocRefMethodWithParametersWithoutSpacesWithHyphens]%
			\StrSubstitute{\jdocRefMethodWithParametersWithoutSpacesWithHyphens}{)}{-}[\jdocRefMethodWithParametersWithoutSpacesWithHyphens]%
			\appto\jdocRefTarget{\#\jdocRefMethodWithParametersWithoutSpacesWithHyphens}%
		}{%
			\appto\jdocRefTarget{\#\jdocRefMethodWithParametersWithoutSpaces}%
		}%
	}%
	%
	\edef\jdocRefLinkText{}%
	\appto\jdocRefLinkText{\jdocRefAtSign}%
	\ifbool{jdocRefShowPackage}{%
		\appto\jdocRefLinkText{\jdocRefPackage}%
		\appto\jdocRefLinkText{.}%
	}{%
	}%
	\appto\jdocRefLinkText{\jdocRefClass}%
	\ifbool{jdocRefShowField}{%
		\IfEq{\jdocRefField}{}{%
		}{%
			\appto\jdocRefLinkText{\#\jdocRefField}%
		}%
	}{%
	}%
	\ifbool{jdocRefShowMethod}{%
		\ifbool{jdocRefShowParameters}{%
			\IfEq{\jdocRefMethodWithParameters}{}{%
			}{%
				\appto\jdocRefLinkText{\#\jdocRefMethodWithoutParameters}%
				\appto\jdocRefLinkText{(}%
				\appto\jdocRefLinkText{\jdocRefParametersText}%
				\appto\jdocRefLinkText{)}%
			}%
		}{%
			\IfEq{\jdocRefMethodWithoutParameters}{}{%
			}{%
				\appto\jdocRefLinkText{\#\jdocRefMethodWithoutParameters}%
				\appto\jdocRefLinkText{(}%
				\appto\jdocRefLinkText{)}%
			}%
		}%
	}{%
	}%
	%
	\IfEq{\jdocRefFullText}{}{%
		\ifbool{jdocRefTypewriter}{%
			\href{\jdocRefTarget}{\texttt{\jdocRefLinkText}}%
		}{%
			\href{\jdocRefTarget}{\jdocRefLinkText}%
		}%
	}{%
		\href{\jdocRefTarget}{\jdocRefFullText}%
	}%
}

\newcommand{\jdocRefPrintAll}{%
	\begin{description}
		\item[BaseUrl] \foreach \sfx in {BaseUrl}{\csname jdocRef\sfx\endcsname}
		\foreach \sfx in {Full, AtSign, Module, MethodWithParameters, MethodWithParametersWithoutSpaces, MethodWithoutParameters, Field, Package, PackageSlashes, Class}{\item[\sfx] \csname jdocRef\sfx\endcsname}
	\end{description}%
}

\DeclareExpandableDocumentCommand{\jdocRefSevenBaseUrl}{}{https\string://docs.oracle.com/javase/7/docs/api/}
\DeclareExpandableDocumentCommand{\jdocRefEightBaseUrl}{}{https\string://docs.oracle.com/javase/8/docs/api/}
\DeclareExpandableDocumentCommand{\jdocRefNineBaseUrl}{}{https\string://docs.oracle.com/javase/9/docs/api/}
\DeclareExpandableDocumentCommand{\jdocRefTenBaseUrl}{}{https\string://docs.oracle.com/javase/10/docs/api/}
\DeclareExpandableDocumentCommand{\jdocRefElevenBaseUrl}{}{https\string://docs.oracle.com/en/java/javase/11/docs/api/}
\DeclareExpandableDocumentCommand{\jdocRefTwelveBaseUrl}{}{https\string://docs.oracle.com/en/java/javase/12/docs/api/}
\DeclareExpandableDocumentCommand{\jdocRefEESevenBaseUrl}{}{https\string://docs.oracle.com/javaee/7/api/}
\DeclareExpandableDocumentCommand{\jdocRefEEEightBaseUrl}{}{https\string://javaee.github.io/javaee-spec/javadocs/}

\DeclareExpandableDocumentCommand{\jdocRefBaseUrl}{}{\jdocRefTwelveBaseUrl}
\DeclareExpandableDocumentCommand{\jdocRefParametersStyle}{}{parenthesis}

%TODO Let travis compile LaTeX. Make sure it fails iff compilation fails. Take https://tex.stackexchange.com/a/48613 (perhaps consider https://tex.stackexchange.com/questions/12067/whats-the-folk-lore-on-automatic-testing-of-tex-programming) for writing integration tests. In a test folder, define a test .tex file. Read https://tex.stackexchange.com/a/161982 and improve code.

%Standard HTML RenderKit : https://docs.oracle.com/javaee/7/javaserver-faces-2-2/renderkitdocs/HTML_BASIC/javax.faces.Inputjavax.faces.Text.html
%\newcommand{\jsftag}[2][]{\jjdocRef{base url = https://docs.oracle.com/javaee/7/javaserver-faces-2-2/vdldocs-facelets/, full, prefix=:, #1}}


